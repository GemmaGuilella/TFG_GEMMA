Pel que fa a la realització d'aquest TFG, s'ha assolit el propòsit principal
de realitzar una aplicació per la gestió automatitzada d'un pàrquing
inte\l.ligent amb el seu control d'entrades i sortides, amb la part de pagament i un panell administratiu.

Per dur a terme aquest projecte s'ha trobat algunes dificultats, entre les quals es poden destacar:

Primer de tot, amb la part de \emph{back-end}:
\begin{itemize}
    \item Crear la composició del codi QR i buscar la forma de què aquests codis siguin vàlids.
    \item Com ha d'actuar el servidor quan rep la lectura d'un codi QR i les accions que ha de
    prendre respecte si és vàlid o no.
    \item En el sistema de pagament, la dificultat és passar-li \emph{metadata} per
    obtenir la informació de qui havia fet el pagament evitant passar les dades per l'URL. D'aquesta manera s'evita que
    l'usuari pugui modificar-la.
    \item Decidir com associar les places als usuaris, és a dir de donar la possibilitat a l'usuari a escollir la plaça
    desitjada o que l'aplicació faci aquesta assignació.
\end{itemize}

Per l'apartat de \emph{front-end} la utilització dels \emph{hooks} que disposa la llibreria
de \emph{React}. Aquests són el \emph{UseCallback}, \emph{useMemo}, \emph{useState} i el \emph{useContext}.
L'explicació d'aquestes eines es poden trobar a \autocite{hooks_react}.

Finalment, per l'apartat de la \emph{Raspberry Pi}:
\begin{itemize}
    \item Buscar eines o llibreries per obrir la càmera de la Raspberry pi.
    \item Buscar llibreries per la detecció als codis QR i la lectura d'aquests.
    \item Buscar una eina per fer peticions HTTP.
    \item Realitzar una màquina d'estats més robusta i que contempli possibles casos d'error, com el d'invalidesa del codi QR.
    \item Buscar components electrònics per tal de poder fer una simulació d'una barrera de pàrquing.
\end{itemize}

Finalment, treballar totes aquestes aplicacions conjuntament i poder fer una demostració de la seva funcionalitat
ha comportat problemes, a causa de la necessitat de controlar la Raspberry Pi per \emph{ssh} i que tots dos conjunts obtinguessin internet.

\section{Treball futur}

Com a treball futur es proposen algunes millores al projecte per tal d'assolir més funcionalitats:

\begin{enumerate}
    \item Afegir o crear un control per part d'administració que pugui pujar o baixar
    la barrera depenent de les necessitats o problemes que puguin succeir. Exceptuant una
    fallada de motor. Per aconseguir aquest efecte és fer que l'actual servidor,
    és a dir, \emph{Laravel} (o \emph{back-end}), passi a ser un client i la Raspberry Pi passi a ser
    un servidor de \emph{Python}. Fer una connexió bidireccional.
    A més a més, crear una taula a la base de dades on es guarda la \emph{ID} i l'estat de la barrera.
    \item Afegir un espai on els usuaris puguin canviar la plaça assignada i puguin
    escollir la plaça desitjada. En l'aplicació fer que els usuaris vegin un mapa
    del pàrquing on pugui seleccionar la plaça, on també mostra les places ocupades
    i lliures en temps real.
    \item Posar la possibilitat de fer reserves de places a través de l'aplicació.
    \item Fins ara, el control d'entrades, sortides i pagaments la veuen els administradors.
    Afegir una pantalla on els clients puguin veure aquesta informació i descarregar-se la factura.
    \item Configurar el servidor de \emph{Laravel} per enviar correus als clients adjuntant la factura
    de cada pagament.
\end{enumerate}
