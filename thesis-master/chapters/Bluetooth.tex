La comunicació amb Bluetooth entre una aplicació
i un dispositiu que serà co\l.locat en la barrera del pàrquing. Aquest dispositiu s'utilitza
per al control d'entrada i sortida d'un garatge privat o d'un
pàrquing d'una institució on els treballadors tenen accés a aquest servei.

\subsubsection{Carecterístiques}
\label{sssec:alternatives}

\begin{itemize}
    \item Obtenir el dispositiu i co\l.locar-lo en la barrera del pàrquing.
    \item Una aplicació que serà el comandament del garatge o el tiquet per entrar al pàrquing.
    \item Donar accés a altres persones.
    \item Tenir un registre d'entrades i sortides.
    \item Comunicació amb Bluetooth encriptat.
\end{itemize}

\subsubsection{Funcionament}

El Bluetooth és un protocol de comunicació que ens permet la transmissió
de dades entre diferents dispositius que es troben a poca distància. La seva transmissió
sense fils es fa través d'ones de ràdio. Per aquesta raó els dispositius han d'estar
lleugerament a prop. Tot i això, podem descobrir diferents dispositius que es classifiquen
segons la seva potència i de la distància on són detectats.


\subsubsection{Conclusió}

Per acabar, el Bluetooth és un dispositiu inte\l.ligent on tindria un paper important
a l'hora d'usar-lo en els garatges o en els pàrquings. Actualment, ja existeixen alguns dispositius
que fan aquest servei, ja que s'insta\l.la un dispositiu junt amb la barrera d'accés i a través aquesta
tecnologia es connecta amb el dispositiu mòbil, que aquest actua com a comandament per poder entrar o sortir
d'aquest. Una alternativa als codis QR utilitzada a garatges privats, o a pàrquings d'institucions privades.