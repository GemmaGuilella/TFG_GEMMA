
Com a treball futur es proposen algunes millores al projecte per tal d'assolir més funcionalitats:

\begin{enumerate}
    \item Afegir o crear un control per part d'administració que pugui pujar o baixar
    la barrera depenent de les necessitats o problemes que puguin succeir. Exceptuant una
    fallada de motor. Per aconseguir aquest efecte és fer que l'actual servidor,
    és a dir, \emph{Laravel} (o \emph{back-end}), passi a ser un client i la Raspberry Pi passi a ser
    un servidor de \emph{Python}. Fer una connexió bidireccional.
    A més a més, crear una taula a la base de dades on es guarda la \emph{ID} i l'estat de la barrera.
    \item Afegir un espai on els usuaris puguin canviar la plaça assignada i puguin
    escollir la plaça desitjada. En l'aplicació fer que els usuaris vegin un mapa
    del pàrquing on pugui seleccionar la plaça, on també mostra les places ocupades
    i lliures en temps real.
    \item Posar la possibilitat de fer reserves de places a través de l'aplicació.
    \item Fins ara, el control d'entrades, sortides i pagaments la veuen els administradors.
    Afegir una pantalla on els clients puguin veure aquesta informació i descarregar-se la factura.
    \item Configurar el servidor de \emph{Laravel} per enviar correus als clients adjuntant la factura
    de cada pagament.
\end{enumerate}
