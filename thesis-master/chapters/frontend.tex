La implementació del \emph{front-end} de l'aplicació, és a dir
la part que l'usuari interactua amb ella s'ha usat la \emph{framework} de \emph{NextJS}
\autocite{nextjs} basada en React.

\section{Elecció de l'entorn}

\emph{React}, és una llibreria de codi obert de JavaScript amb l'objectiu de desenvolupar interfícies d'usuari.
ReactJS permet als desenvolupadors crear aplicacions que fan servir dades que poden canviar
amb el temps sense la necessitat de recarregar la pàgina \autocite{react}. Les seves grans característiques
són la rapidesa, simplicitat i l'escalabilitat. Aquests objectius s'aconsegueixen gràcies a:

\begin{itemize}
    \item \emph{DOM Virtual} \emph{Document Object Model}: és el que permet renderitzar només
    els elements gràfics que han estat modificats, això aporta una gran velocitat. És
    útil quan es té moltes dades i hi ha petites modificacions.
    \item \emph{JSX}: és una extensió del llenguatge de programació JavaScript optimitzat
    en velocitat i captura errors en temps de compilació. L'usuari té la possibilitat
    d'utilitzar-la perquè ajuda al desenvolupador, ja que té una sintaxi semblant a la de
    l'HTML, això implica que fa el codi més llegible i escriure'l és una experiència
    similar a l'HTML, el fa més fàcil.
    \item Permet crear components reutilitzables.
\end{itemize}

Existeixen algunes \emph{frameworks} que usen la llibreria de \emph{React}, com per exemple,
\emph{NextJS} que fa servir tot el que dóna la llibreria de \emph{React}. A més, dóna un sistema
de rutes amb la separació de codi per ruta. Permet en la utilització d'eines que ajuden al
desenvolupador. Algunes d'elles utilitzades són:

\begin{itemize}
    \item \emph{Router}, per l'encaminament i la navegació.
    \item \emph{Link}, un component que permet que l'aplicació enllaci pàgines i carregui les
    seves dades.
\end{itemize}

El més a destacar és que permet la generació estàtica dels arxius, és a dir, precompila part
del codi que no és dinàmic.

\section{Pàgines}
En els annexos s'hi pot trobar un recull de fotografies per mostrar com és la part de \emph{front-end}.
Pel disseny de l'aplicació s'utilitza la \emph{framework} de CSS \texttt{Tailwind}. Usant
aquesta eina permet al desenvolupador evitar escriure arxius CSS \autocite{tailwind}.

\subsection{Iniciar Sessió}
En aquesta vista permet a l'usuari obrir la sessió posant el seu correu electrònic i la contrasenya
en els \emph{inputs} que es veuen a la imatge \autoref{fig:login_photo}. Seguidament, l'usuari ha
de prémer el botó d'\texttt{Iniciar sessió}. Si l'usuari no està registrat en aquesta aplicació, ho
 pot fer prement el botó de \emph{Crear un nou compte} que està sota del primer botó.

\subsection{Registrar usuari}
Aquesta vista mostra uns \emph{inputs} per incloure les dades de l'usuari per
poder-se registrar a l'aplicació. Aquests són el nom, el DNI, el telèfon mòbil, el correu electrònic, la contrasenya
i la confirmació de la contrasenya. Una vegada inscrites aquestes dades s'ha de prémer el botó de \emph{Registrar-se},
si es vol tirar enrere i anar a la pàgina d'iniciar sessió es pot clicar el botó de \emph{Ja tinc compte} o prémer el \emph{logo} de
la \emph{navbar}. La imatge relacionada aquesta vista es pot trobar a \autoref{fig:register_photo}.

\subsection{Principal}
La imatge que mostra un exemple d'aquesta vista està a \autoref{fig:index_photo}. En la fotografia es pot veure tres seccions.

La primera és que dóna la benvinguda a l'usuari i hi ha un botó que únicament es pot clicar quan està disponible.
La disponibilitat varia segons si l'usuari és administrador. La imatge corresponent a la part d'administradors
està a \autoref{fig:admin_photo}.

La segona, hi ha una \emph{card} on hi ha la informació de l'usuari, a la part superior a la dreta
hi ha una icona d'un llapis. Aquest llapis és un botó per editar aquesta informació. Quan l'usuari
el clica, l'aplicació redirigeix a l'usuari a una pantalla per editar-la. La pantalla d'editar està a
\autoref{fig:edit_user_photo}.

Finalment, es troba la tercera secció on també hi ha una \emph{card} on hi ha la informació dels vehicles
registrats de l'usuari. A la part superior d'aquesta \emph{card} hi ha un botó amb un \texttt{+},
si el client clica aquest botó, l'aplicació el redirigeix a una pantalla
per afegir vehicles, una imatge d'aquesta vista està a \autoref{fig:add_car_photo}.

Quan hi ha vehicles guardats es mostra el número de matrícula i tres botons. Aquests botons tenen tres
accions diferents, el primer canvia depenent d'on està el cotxe. Si el cotxe no està dins del pàrquing,
el botó és un codi QR, ja que per entrar al pàrquing s'ha de clicar. Es pot veure aquesta pantalla a
\autoref{fig:codiQR_photo}. En canvi, si el cotxe està dins del pàrquing la vista mostra una \texttt{P}
de pàrquing, on hi ha la informació relacionada d'aquest aparcament.
Un exemple d'aquesta pantalla és \autoref{fig:information_photo}.
El segon botó és un llapis, s'utilitza per editar el número de la matrícula.

La vista que controla aquesta funcionalitat és a \autoref{fig:edit_car_photo}. Per acabar, hi ha un botó
d'una brossa de color vermell. Aquest botó és per eliminar el vehicle.
Els botons d'editar i eliminar es deshabiliten quan el vehicle està estacionat dins del pàrquing.

\subsubsection{Vista per editar l'usuari}
En aquesta vista l'usuari pot modificar les dades del seu compte com el nom, la contrasenya, el telèfon i el DNI.
Es troben dos botons al final de la \emph{card} on el primer és per tirar enrere cap a la pantalla d'índex
\autoref{fig:index_photo}, també es pot anar a la pantalla principal clicant el \emph{logo} de la \emph{navbar}.
El segon botó és per guardar les noves dades a la base de dades. La imatge relacionada a aquesta vista és \autoref{fig:edit_user_photo}.

\subsubsection{Vista per editar els cotxes de l'usuari}
La imatge que mostra aquesta pantalla és \autoref{fig:edit_car_photo}. Es veu una \emph{card} on només hi ha un \emph{input}
per modificar el número de la matrícula.
El format pel número de la matrícula és \emph{xxxxXXX}, on els quatre primers caràcters són números del 0-9
i els tres caràcters del final són lletres de l'A-Z. La matrícula és única per cada usuari, és a dir, no es pot repetir.

\subsubsection{Vista per afegir els cotxes de l'usuari}
Aquesta vista és molt similar a l'anterior, ja que únicament canvia el títol i l'acció resultant,
és a dir afegir un cotxe a la base de dades amb l'usuari que ha iniciat sessió.
La següent imatge mostra aquesta vista \autoref{fig:add_car_photo}.

\subsection{Codi QR}
Quan l'usuari es troba a la pàgina principal i vol accedir amb un vehicle al pàrquing, clica el botó del codi QR que pertany en aquest vehicle.
Aquest botó redirigeix a aquesta pantalla \autoref{fig:codiQR_photo}. Es veu una \emph{card} on a dins hi ha el codi QR
generat per aquell vehicle. A sota d'aquest codi QR hi ha un botó, aquest botó serveix per generar un altre codi QR. Aquest botó
s'utilitza quan el codi QR generat no és vàlid
i es necessita crear-ne un altre per accedir dins del pàrquing.

A sota hi ha dos botons, el botó d'\emph{Enrere} per anar a la pantalla d'inici, també es pot usar el \emph{logo}
de la \emph{navbar} que redirigeix a la mateixa pàgina principal. A la dreta es pot veure un botó on diu \emph{Informació}
que s'activa quan el codi QR ha estat validat pel servidor i es pot entrar al pàrquing. En aquesta es veu la informació de l'estacionament.
Aquest botó, en principi no se'n fa ús, ja que el servidor envia un esdeveniment a l'aplicació a temps real i redirigeix
a la pàgina d'informació. Es pot veure un exemple a \autoref{fig:information_photo}.
Tot i això, aquest botó és útil per si hi ha algun problema que impedís l'enviament d'aquest esdeveniment.
D'aquesta manera l'aplicació es pot continuar fent servir.

\subsection{Informativa}
Aquesta pantalla ens mostra la informació de quina plaça té associada el cotxe, el dia i hora que ha entrat al pàrquing,
el preu en hora, la data actual i el preu en temps real. És una vista informativa. Quan es vol sortir del
pàrquing s'ha de prémer el botó de sortir que hi ha al final de la \emph{card} on es redirigeix a la pantalla
de pagament que es pot veure a \autoref{fig:payment_photo}. La següent imatge és un exemple del que pot mostrar aquesta pantalla
informativa \autoref{fig:information_photo}.

\subsection{Pagament}
Un cop s'ha pres el botó de sortir de la pantalla anterior es redirigeix a una pantalla de \emph{stripe}
per fer el pagament. Per simular un pagament amb èxit existeixen una llista de targetes
per fer \emph{testing}. Aquesta llista es pot trobar a \autocite{list_cards}.
Un exemple d'aquesta pantalla de pagament de \emph{Stripe} \autocite{stripe} està a \autoref{fig:payment_photo}.

\subsubsection{Pagament validat}
La pantalla de quan el pagament és validat consta d'un \emph{tick} de color verd
per mostrar que tot ha funcionat correctament. Un codi QR generat per sortir del pàrquing amb
un text explicatiu per saber que ha anat tot correctament i de quant temps es disposa
per la validesa d'aquest codi QR. Si aquests minuts s'accedeixen, s'ha de tornar a pagar.
Disposa d'un botó per anar a la pàgina principal, com també si es clica el \emph{logo} de la
\emph{navbar}. La vista es pot veure a \autoref{fig:payment_success_photo}.

\subsubsection{Pagament cence\l.lat}
En aquest cas, el pagament ha estat cance\l.lat per l'usuari. La vista és
una creu de color vermell que indica que hi ha hagut un error, a més una descripció posa que el pagament
no s'ha efectuat. Hi ha un botó per anar a la pàgina principal, com també si es clica el \emph{logo} de la
\emph{navbar}. La imatge d'aquesta vista es pot veure a \autoref{fig:payment_cancel_photo}

\subsection{Panell administració}
\label{sssec:index}
En aquest panell d'administració només hi pot accedir els usuaris que són \emph{admin}.
Els altres usuaris no hi poden accedir i a més tenen el botó deshabilitat.
Com es pot observar en la imatge d'aquesta vista a \autoref{fig:admin_photo} hi ha
dues \emph{cards}. La primera l'administrador pot modificar el preu en hores del
pàrquing, s'ha de posar en cèntims i també pot modificar el temps de validesa dels
codis QR. La segona hi ha una taula on es mostren tots els \emph{logs}, és a dir, totes
les entrades, sortides i pagaments efectuats al pàrquing. A la columna de \emph{pagament}
hi ha un enllaç cap a \emph{stripe} on hi ha la informació del pagament.
En aquesta pantalla es pot fer accions com descarregar les factures, tramitar devolucions,
veure detalls de la transacció i client \autocite{stripe}.
La imatge es pot veure a \autoref{fig:stripe_info_photo}.

\section{Eines utilitzades}

En aquesta aplicació es fan servir diferents eines importants per fer un correcte funcionament.
Aquestes eines són:

\begin{itemize}
    \item \texttt{Axios} \autocite{axios}: Aquesta eina permet fer peticions HTTP passant-li en els \emph{Authoritzon Header} el \emph{Barear Token} de l'usuari.
    El fet aquest es pot configurar de manera automàtica si s'activa les \emph{withCredentials} d'axios o es pot passar manual.
    \item \texttt{classnames} \autocite{classnames}: Una senzilla utilitat de JavaScript per unir condicionalment classNames.
    \item \texttt{dayjs} \autocite{dayjs}: Una llibreria de JavaScript de manipulació de dates.
    \item \texttt{pusher-js} \autocite{pusher}: Els canals de Pusher són una solució als \emph{Web Sockets}.
    Eina útil per crear aplicacions interactives en temps real.
    \item \texttt{react-qr-codes} \autocite{react_qr_code}: Un component de React per crear codis QR.
    \item \texttt{swr} \autocite{swr}: Utilitat que permet evitar peticions HTTP innecessàries utilitzant una \emph{cache}.
\end{itemize}
